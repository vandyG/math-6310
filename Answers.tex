\documentclass{amsart}[11pt]
\usepackage[utf8]{inputenc}

\pagestyle{plain}

\usepackage{hyperref}
%\usepackage{tablefootnote}
\usepackage{graphicx}
\usepackage{subcaption}
\usepackage{amsthm}
\usepackage{dsfont}
\usepackage{amssymb}
\usepackage{enumerate}
\usepackage{graphicx}
\usepackage{mathrsfs}
\usepackage{float}
\usepackage{bbm}
\usepackage{amsmath}
\usepackage{comment}
\usepackage{hyperref}
\usepackage{listings}
\usepackage{color}
\usepackage{ulem}
\usepackage[dvipsnames]{xcolor}
\usepackage[lined,boxed,commentsnumbered]{algorithm2e}
\usepackage{fourier}
\usepackage{multicol}
\usepackage{soul}
\allowdisplaybreaks

\hypersetup{
	colorlinks,
	citecolor=blue,
	filecolor=blue,
	linkcolor=blue,
	urlcolor=blue,
	%hyperfootnotes=false
}

\newtheorem{theorem}{Theorem}[section]
\newtheorem{proposition}[theorem]{Proposition}
\newtheorem{lemma}[theorem]{Lemma}
\newtheorem*{lem}{Lemma}
\newtheorem{corollary}[theorem]{Corollary}
\newtheorem{definition}[theorem]{Definition}

\newtheorem{claim}[theorem]{Claim}
\newtheorem{thmnum}{Theorem}
\renewcommand{\thethmnum}{\Alph{thmnum}}
\newtheorem{propnum}{Proposition}
\renewcommand{\thepropnum}{\Alph{propnum}}
\newtheorem{cornum}{Corollary}
\renewcommand{\thecornum}{\Alph{cornum}}

\newtheorem*{theoremos}{Theorem}
\newtheorem*{prop}{Proposition}
\theoremstyle{definition}
\newtheorem{example}[theorem]{Example}
\newtheorem{question}{Question}
\newtheorem{problem}{Problem}
\newtheorem{solution}{Solution}
\newtheorem{experiment}{Experiment}
\newtheorem{remark}[theorem]{Remark}

% % % % % % % % % % %our macros % % % % % % % % % % % % % % %
\newcommand{\esssup}{{\mathrm{ess}\sup}}
\newcommand{\cF}{\mathcal{F}}
\newcommand{\cT}{\mathcal{T}}
\newcommand{\cS}{\mathcal{S}}
\newcommand{\TT}{\mathbb{T}}
\newcommand{\cV}{\mathcal{V}}
\newcommand{\cX}{\mathcal{X}}
\newcommand{\HH}{\mathcal{H}}
\newcommand{\Shah}{{\makebox[2.3ex][s]{$\sqcup$\hspace{-0.15em}\hfill $\sqcup$}  }}
\newcommand{\cA}{\mathcal{A}}
\newcommand{\bc}{\textbf{c}}
\newcommand{\Wcp}{{W_0(L^p)}}
\newcommand{\lsp}{{\boldsymbol\ell}}
\newcommand{\Wco}{{W_0(L^1)}}
\newcommand{\R}{\mathbb{R}}
\newcommand{\Z}{\mathbb{Z}}
\newcommand{\N}{\mathbb{N}}
\newcommand{\C}{\mathbb{C}}
\newcommand{\K}{\mathbb{K}}
\providecommand{\abs}[1]{\lvert#1\rvert}
\providecommand{\norm}[1]{\lVert#1\rVert}
\newcommand{\G}{\mathcal{G}}
\newcommand{\T}{\mathcal{T}}
\newcommand {\D} {\mathbb D}
\newcommand{\E}{\mathbb{E}}
\newcommand{\Prob}{\mathbb{P}}
\newcommand{\sspan}{\textnormal{span}}
\newcommand{\bracket}[1]{\left\langle#1\right\rangle}
\newcommand{\col}{\textnormal{Col}}
\newcommand{\row}{\textnormal{Row}}
\newcommand{\rank}{\textnormal{rank}}
\newcommand{\diag}{\textnormal{diag}}
\newcommand{\tr}{\textnormal{tr}}
\newcommand{\fro}{\textnormal{F}}
\newcommand{\rw}{\textnormal{rw}}
\newcommand{\sym}{\textnormal{sym}}
\newcommand{\eps}{\varepsilon}

\allowdisplaybreaks

\title{MATH 6310 (Fall 2025) Homework Solutions}
\author{Vandit Goel\\\textbf{Updated:} Wednesday  3 September, 2025}

\begin{document}

\maketitle

\section{Linear Algebra Problems}

\begin{problem}
If $A\in\R^{n\times n}$ is symmetric, invertible, and has spectral decomposition $A=V\Lambda V^{-1}$, give (with proof) a formula for $A^{-1}$?

\begin{solution}
Let $A \in \mathbb{R}^{n \times n}$ be symmetric and invertible, with spectral decomposition
\[
A = V \Lambda V^{-1},
\]
where $V$ is an orthogonal matrix ($V^{-1} = V^\top$) and $\Lambda$ is diagonal.

\vspace{\baselineskip}
\noindent To find $A^{-1}$, we use the property that for invertible matrices:
\begin{align*}
	(A)^{-1} &= (V \Lambda V^{-1})^{-1} \\
	&= (V^{-1})^{-1}\Lambda^{-1} V^{-1} \\
	&= V\Lambda^{-1} V^{-1}
\end{align*}
Thus, the inverse is given by
\[
A^{-1} = V \Lambda^{-1} V^{-1}.
\]
\end{solution}
\end{problem}


\begin{problem}
\begin{enumerate}[(a)]
    \item Prove that $\mathcal{N}(A^\top A)=\mathcal{N}(A)$
    \item Prove that $\mathcal{N}(A)$ is orthogonal to $\row(A)$
    \item How does $\mathcal{N}(AB)$ relate to $\mathcal{N}(A)$ and $\mathcal{N}(B)$? Prove any statements you claim.
    \item Use Sylvester's Inequality to prove that if $A\in\R^{m\times r}$ and $B\in\R^{r\times n}$ have rank $r$, then $\rank(AB)=r$.
\end{enumerate}
\begin{solution}
    \begin{enumerate}[(a)]
    \item 
    \end{enumerate}
\end{solution}
\end{problem}


\end{document}

